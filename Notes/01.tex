\chapter{Intro to naive set theory}

\begin{definition}
  We call $B$ is a \textbf{subset} of a set $A$, $B \subseteq A$ if for every $x \in B$, $x \in A$.

  We call $B$ is a \textbf{proper subset} of a set $A$, $B \subsetneq A$ if $B \subseteq A$ and $B \neq A$.
\end{definition}

\begin{definition}
  empty set denoted as $\emptyset$ 
\end{definition}

\begin{definition}
  Given $A$, a set,  
  \[P(A) = \{B \mid B \subseteq A\}\]
  called \textbf{power set} of $A$.
\end{definition}

\section{Cardinality}
\begin{definition}
  Given a set $A, B$, then if there exists bijection from $A$ to $B$,
  then $A$ and $B$ have the same \textbf{cardinality}, denoted as $|A| = |B|$.
\end{definition}

\begin{definition}
  If there is an injection from $A$ to $B$ then we write $|A| \le |B|$.
If $|A| \le |B|$ and $|A| \neq |B|$ then we write $|A| < |B|$.
\end{definition}

\begin{theorem}
  Let $A$ and $B$ be two sets, then either $|A| \le |B|$, or $|B| \le |A|$.
Furthermore, if $|A| \le |B|$ and $|B| \le |A|$ then $|A| = |B|$.
\end{theorem}

\begin{theorem}
  Let $A$ be a set, then $|A| < |P(A)|$.
\end{theorem}

\subsection{Finite set and infinite set}

\begin{theorem}
Let $A$ be a set. The followings are equivalent:
\begin{enumerate}
  \item There is a bijection between $A$ and a proper subset of $A$.
  \item There is an injection from $\NN$ to $A$.
  \item There isn't a bijection from $A$ to $\{x \in \NN \mid x < n\}$.
\end{enumerate}
\end{theorem}

\begin{definition}
  If a set A satisfies any of the 3 conditions in the theorem
above we call it an \textbf{infinite set}. Otherwise we call it a \textbf{finite set}.
\end{definition}

\subsection{Countable infinite set}

\begin{definition}
  If $|A| = |\NN|$, we call $A$ a \textbf{countable infinite set}. If
  $|A| > |\NN|$ we call $A$ an \textbf{uncountable set}.
\end{definition}


