\chapter{Metric Spaces}
\section{Definition}
\begin{definition}
  $(X, d), d: X \times X \to [0, \infty)$ is a metric space if: $\forall x,y,z \in X$
  \begin{itemize}
    \item $d(x,y) = 0 \iff x = y$
    \item $d(x,y) = d(y,x)$
    \item $d(x,y) \leq d(x,z) + d(z,y)$
  \end{itemize}
\end{definition}

\begin{theorem}
  Given $(X, d)$ metric space, $(X, f \circ d)$ is a metric space if
  \begin{itemize}
    \item $f(x) = 0 \iff x = 0$
    \item $f(x+y) \leq f(x) + f(y)$
    \item $f$ is non-decreasing
  \end{itemize}
\end{theorem}

\begin{theorem}
  Let $(X, d_x)$ and $(Y, d_y)$ be metric spaces.
  Then $(X \times Y, d_{\sup})$ is a metric space where
  \[d_{\sup}((x, y), (x', y')) = \max(d_x(x, x'), d_y(y, y'))\]
\end{theorem}

\begin{theorem}
  Given $(X, d)$ metric space, and $S$ be non-empty set. Let
  \[Y = \{f \in Map(S, X) \mid diam(f(S)) < \infty \}\]
  and 
  \begin{align*}
    d_{\sup}: Y \times Y &\to [0, \infty)\\
    f, g&\mapsto \sup\{d(f(S), g(S)) \mid s \in S\}
  \end{align*}
  Then $(Y, d_{\sup})$ is a metric space.
\end{theorem}

\section{Open sets and closed sets}
\subsection{Definition}
\begin{definition}[Open Ball]
  Let $(X, d)$ be a metric space.
  \[B_X(x, r) = \{y \in X \mid d(x,y) < r\}\]
  is called an open ball centered at $x$ with radius $r$.
\end{definition}
\begin{definition}[Open Set]
  Let $(X, d)$ be a metric space.
  $A \subseteq X$ is called an open set if 
  \[\forall a \in A, \exists r > 0, B_X(a, r) \subseteq A\]
\end{definition}
\begin{definition}[Closed Set]
  Let $(X, d)$ be a metric space.
  $A \subseteq X$ is called an closed set if 
  \[\forall a \in X\setminus A, \exists r > 0, B_X(a, r) \cap A = \emptyset\]
  (or equivalently, $X\setminus A$ is an open set)
\end{definition}

\begin{theorem}
  Any open ball is open
\end{theorem}

\begin{theorem}
  Any closed ball is closed.
  Furthermore any set consisting of a single point is closed.
\end{theorem}

\subsection{Basic properties}
\begin{theorem}
  The union of a set of open sets is open.
\end{theorem}

\begin{definition}
  Let $(X, d)$ be a metric space, $A \subseteq X$. We define:
  \begin{itemize}
    \item The \textbf{closure} of $A$ as
    \[
      \overline{A}= \bigcap\{V \subseteq X \mid V \text{ closed }, A \subseteq V \}
    \]
    \item The \textbf{interior} of $A$ as
    \[
      A^\circ = \bigcup\{U \subseteq X \mid U\text{ open }, U \subseteq A\}
    \]
    \item The \textbf{boundary set} of $A$ is
    \[
      \partial A = \overline{A} \setminus A^\circ
    \]
\end{itemize}
\end{definition}

\begin{theorem}
  The intersection of finitely many open sets is open
\end{theorem}

\begin{theorem}
  A non-empty subset of a metric space $(X, d)$ is open $\iff$ it is
a union of open balls
\end{theorem}

\subsection{Openness in subspaces}

\begin{theorem}
  If $(X, d)$ is a metric space, $(Y, d|_{Y \times Y} )$ a subspace.
  \begin{itemize}
    \item $A \subseteq Y$ is open in $Y$ $\iff$ there is some open set $A'$ in $X$ where $A = A' \cap Y$
    \item $A \subseteq Y$ is closed in $Y$ $\iff$ there is some closed set $A'$ in $X$ where $A = A' \cap Y$
  \end{itemize}
\end{theorem}
\subsection{Denseness}

\begin{definition}
  A subset $A \subseteq X$ is dense if any non-empty open subset of
$X$ has non-empty intersection with $A$.
\end{definition}

\subsection{Open sets and closed sets in $\RR$}

\begin{lemma}
  Let $d_e(x, y) = |x - y|$
  \begin{itemize}
    \item  Any finite open interval $(a, b)$, where $b > a$, is open
    under metric $d_e$.
    \item Any infinite open interval $(-\infty, a), (a, \infty)$ or $(-\infty, \infty)$ must be open under
    metric $d_e$.
  \end{itemize}
\end{lemma}

\begin{theorem}
  If $A \subset \RR$ is non-empty and $sup(A) < \infty$. Then
  \begin{itemize}
    \item If $A$ is open, then $\sup(A) \notin A$.
    \item If $A$ is closed, then $\sup(A) \in A$.
  \end{itemize}
\end{theorem}

\begin{theorem}
  $A \subseteq \RR$, $A \neq \emptyset$, is open under $d(x, y) = |x-y|$ $\iff$ 
  it is the disjoint union of finite or countably infinitely many (finite or infinite) open intervals
\end{theorem}

\section{Continuity}

\subsection{Definition}

\begin{definition}
  Given $(X, d)$ and $(Y, d')$ metric spaces, a function $f: X \to Y$ 
  \begin{itemize}
    \item is called continuous if for any open set $U \subseteq Y, f^{-1}(U)$ is open in $X$
    \item is called continuous at $x \in X$ if for any open set $U \subseteq Y, f(x) \in U$,
    there is an open set $V \subseteq X, x \in V$ such that $f(V) \subseteq U$ 
  \end{itemize}
\end{definition}

\subsection{Localness and $\eps-\delta$ characterization}

\begin{theorem}
  $f: X \to Y$ is continuous $\iff$ for any $x \in X$, $f$ is continuous at $x$.
\end{theorem}

\begin{theorem}
  $f: X \to Y$ is continuous at $x \in X$ $\iff$ for any $\eps > 0$,
  there exists $\delta > 0$ such that 
  \[f(B_X(x, \delta)) \subseteq B_Y(f(x), \eps)\]
\end{theorem}

\begin{theorem}
  A map $f$ from $(X, d)$ to $(Y, d')$ is continuous if for any $x, y \in X$,
  for any $\eps > 0$, there exists $\delta > 0$ such that 
  \[d(x, y) < \delta \implies d'(f(x), f(y)) < \eps\]
\end{theorem}

\begin{definition}
  If a bijection between two metric space and its inverse are
both continuous, we call it a \textbf{homeomorphism}
\end{definition}

\subsection{Real valued continuous functions}

\begin{theorem}[Intermediate Value Theorem]
  if $f: [a, b] \to \RR$ is continuous map,
  $m \in \RR, (f(a) - m)(f(b) - m) < 0$ then there is some 
  $c \in (a, b)$ such that $f(c) = m$
\end{theorem}

\section{Compactness}

\begin{definition}
  A metric space $X$ is called \textbf{compact}, if for every set of open subset 
  $C$ that $\bigcup C = X$, there is a finite subset $C'$ such that $\bigcup C' = X$
\end{definition}

\begin{theorem}
  $(X, d)$ is compact $\iff$ $\forall C \subseteq \{B_X(x, r) \mid x \in X, r \in (0, \infty)\}$, 
  $\bigcup C = X \implies \exists C' \subseteq C$ such that $\bigcup C' = X$ and $C'$ is finite
\end{theorem}

\begin{theorem}
  Any compact metric space has finite diameter ($diam(X) < \infty$)
\end{theorem}

\begin{theorem}
  If $(X, d)$ is a metric space, $Y \subseteq X$, a non-empty subset, with subspace metric. Then, 
  $(Y, d|_{Y \times Y})$ is compact $\implies$ $Y$ is closed.
\end{theorem}

\begin{theorem}
  If $(X, d)$ is a compact metric space $V \subseteq X$ is closed, then $V$ under subspace metric is also compact
\end{theorem}

\begin{theorem}
  $f: (X, d) \to (Y, d')$ is continuous and surjection. 
  If $X$ is compact, then $Y$ is compact.
\end{theorem}

\begin{theorem}
  If $(X, d), (Y, d')$ are both compact metric spaces, then $(X, Y, d_{\sup})$ is also compact
\end{theorem}

\begin{theorem}
  A subset of $\RR^n$ is compact $\iff$ it is bounded and closed
\end{theorem}
